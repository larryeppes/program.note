\chapter{Linux基础命令}

%%%%%%%%%%%%%%%%%%%%%%%%%%%%%%%%%%%%%%%%%%
%%%%%%%%%%%%%%%%%%%%%%%%%%%%%%%%%%%%%%%%%%
%%%%%%%%%%%%%%%%%%%%%%%%%%%%%%%%%%%%%%%%%%
chroot 在多 Linux 系统下, 在指定根目录下运行指令.
\begin{shell}
sudo mount -t ext3 /dev/sdb1/mnt/
cd /mnt
sudo chroot /mnt/
\end{shell}

arch: 打印当前硬件架构.

df -T: 显示文件系统类型.

info emacs buffers -o info-out.txt: 将 emacs 中 buffers 节点信息存入文件.

bc file: 计算 file 中的任务.

lp: 用于打印文件或修改排队的打印任务. lp (选项) (参数)
\begin{itemize}
 \item[-E:] 打印时强制加密
 \item[-d:] 接收打印任务的目标打印机
 \item[-i:] 指定存在的打印任务号
 \item[-m:] 打印完发e-mail
 \item[-n:] 指定打印份数
 \item[-p:] 打印页码
\end{itemize}

ifconfig: 配置和显示 linux 网络接口的网络参数.
\begin{shell}
## 为网络接口 eth0 设 IP 和掩码
ifconfig eth0 192.168.0.2 netmask 255.255.255.224
\end{shell}

elinks: 文本界面访问 web 页面. elinks (选项) (参数)
\begin{shell}
elinks -dump: html 文档以文本方式到输出流.
\end{shell}

emacs (选项) (参数): 参数指定文本文件, 选项有
\begin{itemize}
 \item[+\textlangle 行号\textrangle] 启动 emacs, 光标到行号处
 \item[-q] 不加载初始文件启动 emacs
 \item[u \textlangle用户\textrangle] 加载指定用户初始化文件启动文件
 \item[-t \textlangle文件\textrangle] 指定文件作终端启动 emacs, 不适用 stdin 和 stdout
 \item[-f \textlangle函数\textrangle] 执行指定 lisp (广泛用于人工智能邻域的编程语言) 函数
 \item[-l \textlangle lisp 代码文件\textrangle] 加载指定 lisp 文件
 \item[-batch] 批处理模式运行 emacs
\end{itemize}

scp: 以加密方式在本地/远程主机和远程/本地主机之间复制文件.

dump: 用于备份 ext2/ext3 文件系统

restore 命令与 dump 互逆, 用于还原备份文件.
\begin{shell}
sudo dump -f boot-dum.bak /boot/
sudo rm -fr /boot/*
cd /boot
sudo restore -tf boot-dump.bak
\end{shell}

cpio: 复制文件到归档包中或从归档包中复制文件.
\begin{shell}
find /etc -print | cpio -o > etc.bak
\end{shell}

tar 主要用于创建和解压归档文件, 其本身没有压缩功能, 但可以调用 gzip, bzip2 进行压缩处理.
参数解释:
\begin{itemize}
 \item[-c] 创建归档
 \item[-x] 解压归档
 \item[-v] 显示处理过程
 \item[-f] 目标文件
 \item[-j] 调用 bzip2 进行解压缩
 \item[-z] 调用 gzip 进行解压缩
 \item[-t] 列出归档中文件
\end{itemize}

useradd 与 passwd
\begin{shell}
## -m 创建 home 目录
## -g 所属的主组
## -G 指定用户在哪些附加组
## -s 设定默认的 shell
## newuser 新的用户名
useradd -m -g users -G audio -s /usr/bin/bash newuser
## 不带参数更改当前用户密码
passwd
## 更改 newuser 的用户密码
passwd newuser
\end{shell}

whereis 用于查找文件, 手册等
\begin{shell}
whereis bash
## 仅查找 binary
whereis -b bash
## 仅查找 manual
whereis -m bash
\end{shell}

find 用于查找文件, 支持正则, 并可将查找结果传递到其他命令
\begin{shell}
## 当前目录查找符合 pattern 的文件
find . -name pattern
## 从 home 目录查找所有符合 pattern 的文件, 并交由 ls 输出详细信息
find /home -name pattern -exec ls -l {} \;
\end{shell}

wget 用于下载
\begin{shell}
## 下载文件并重命名为 newname.md
wget -O newname.md url
## 下载 url 并开启断点续传
wget -c url
\end{shell}


在终端使用powerline, tmux, 可以美化终端. 首先由dnf安装powerline, 
然后在.bashrc文件中添加
\begin{shell}
if [ -f `which powerline-daemon` ]; then
  powerline-daemon -q
  POWERLINE_BASH_CONTINUATION=1
  POWERLINE_BASH_SELECT=1
  . /usr/share/powerline/bash/powerline.sh
fi
\end{shell}
用dnf安装tmux-powerline, 然后编辑\cmd{~/.tmux.conf}添加一行
\begin{shell}
 source "/usr/share/tmux/powerline.conf"
\end{shell}

安装\cmd{vim-plugin-powerline}可以美化vim编辑器的状态栏.

\subsection{Markdown工具}
Markdown是一款文本标记语言, 易写易读. 
\begin{verbatim}
*强调*
# This is an H1
## This is an H2
###### This is an H6
\end{verbatim}
首行的\#字数量决定标题的级别. 无序列表使用星号, 加号, 减号.
\begin{verbatim}
* Red
* Green
* Blue
+ Red
+ Green
+ Blue
- Red
- Green
- Blue
\end{verbatim}
在列表项目内放进引用, 那>就要缩进.
\begin{verbatim}
A list item with a blockquote:
> This is a blockquote
> inside a list item.
\end{verbatim}
代码区块, 只要缩进4个空格或一个制表符即可.

分割线: 一行中用三个以上的星号, 减号, 底线建立, 行内不能有任何其他东西, 星号或减号中间可以插入空格.

用反引号\textasciigrave包起来标记一小段代码.
\begin{itemize}
 \item 语法
 \begin{itemize}
  \item 行内HTML
  \item 特殊字符转换
 \end{itemize}
  \item 区块元素
  \begin{itemize}
   \item 段落, 换行
   \item 标题
   \item 区块引言
   \item 列表
   \item 程序代码块
   \item 分割线
  \end{itemize}
  \item 区段元素
  \begin{itemize}
   \item 链接
   \item 强调
   \item 程序代码
   \item 图片
  \end{itemize}
  \item 其他
  \begin{itemize}
   \item 转义字符
   \item 自动链接
  \end{itemize}
\end{itemize}

\section{Linux C}

%%%%%%%%%%%%%%%%%%%%%%%%%%%%%%%%%%%%%%%%%%
%%%%%%%%%%%%%%%%%%%%%%%%%%%%%%%%%%%%%%%%%%

\subsection{linux GCC/G++编译器与调试器}
GCC是GNU的一个子项目, 最初作为C语言的编译器. 现在GCC能编译C, 
C++, Ada, Object C和Java等语言, 同时还可执行跨硬件平台的交叉编译.

安装GCC和G++的命令如下:
\begin{shell}
 yum install make gcc gcc-c++
\end{shell}

GCC/G++编译选项
\begin{table}[H]
\small % fontsize
\sffamily % font family = sans serif
\centering % alignment of tabular
\renewcommand{\arraystretch}{1.4} % larger cell height
\caption{常用GCC/G++编译选项}
\begin{tabular}{*{2}{l|l}}
 \hline
 \bfseries 编译选项 & \bfseries 说明 \\
 \hline
 -c & 只进行预处理, 编译和汇编, 生成.o文件\\
 \hline
 -S & 只进行预处理和编译, 生成.s文件\\
 \hline
 -E & 只进行预处理, 产生预处理后的结果到标准输出\\
 \hline
 -C & 预处理时不删除注释信息, 常与-E同时使用\\
 \hline
 -o & 指定目标名称, 常与-c, -S同时使用, 默认是.out\\
 \hline
 -include file & 插入一个文件, 功能等同源代码中的\#include\\
 \hline
 -Dmacro[=defval] & 定义一个宏, 功能等同源代码中的\#define macro [defval]\\
 \hline
 -Umacro & 取消宏定义, 功能等同源代码中的\#undef macro\\
 \hline
 -Idir & 优先在选项后的目录中查找包含的头文件\\
 \hline
 -lname & 链接后缀为.so的动态链接库来编译程序\\
 \hline
 -Ldir & 指定编译搜索库的路径\\
 \hline
 -O[0-3] & 编译器优化, 数值越大优化级别越高, 0没有优化\\
 \hline
 -g & 编译器编译时加入编译信息\\
 \hline
 -pg & 编译器加入信息给gprof\\
 \hline
 -share & 使用动态库\\
 \hline
 -static & 禁止使用动态库\\
 \hline
\end{tabular}
\end{table}

%%%%%%%%%%%%%%%%%%%%%%%%%%%%%%%%%%%%%%%%%%
%%%%%%%%%%%%%%%%%%%%%%%%%%%%%%%%%%%%%%%%%%

\subsection{程序和进程}

程序\index{程序}是指一组指示计算机或其他具有信息处理能力设备每一步动作的指令.
进程\index{进程}是一个具有独立功能的程序关于某个数据集合的一次可以并发执行的运行活动, 
是处于活动状态的程序. 
在Linux系统中, 用户创建进程时会先在系统的进程表中为进程创建独一无二的编码,
即PID\index{PID}.

%%%%%%%%%%%%%%%%%%%%%%%%%%%%%%%%%%%%%%%%%%
%%%%%%%%%%%%%%%%%%%%%%%%%%%%%%%%%%%%%%%%%%

\subsection{GDB调试器}
用\cmd*{sudo yum install gdb}安装GDB调试器.
GDB调试器调试的对象是可执行文件, 使用GCC或G++编译器编译源代码时, 
必须加上选项-g才能使目标可执行文件包含被调试的信息. 如
\begin{shell}
 gcc -g -o helloworld helloworld.c // 编译并连接程序, 使之包含可被调试信息
 gdb helloworld // 使用GDB调试器打开 helloworld 可执行文件
\end{shell}

\begin{table}[H]
 \small % fontsize
\sffamily % font family = sans serif
\centering % alignment of tabular
\renewcommand{\arraystretch}{1.4} % larger cell height
\caption{常用GDB命令及解释}
\begin{tabular}{*{2}{l|l}}
 \hline
 \bfseries 命令 & \bfseries 解释\\ \hline
 file [文件名] & 在GDB中打开执行文件\\ \hline
 break & 设置断点, 支持的形式由break 行号, break 函数名, break 行号/函数名 if 条件\\ \hline
 info & 查看和可执行文件相关的各种信息\\ \hline
 kill & 终止正在调试的程序\\ \hline
 print & 显示变量或表达式的值\\ \hline
 set args & 设置调试程序的运行参数\\ \hline
 delete & 删除设置的断点或观测点\\ \hline
 clear & 删除设置在指定行号或函数上的断点\\ \hline
 continue & 从断点处继续执行程序\\ \hline
 list & 列出GDB中打开的可执行文件代码\\ \hline
 watch & 在程序中设置观测点\\ \hline
 run & 运行打开的可执行文件\\ \hline
 next & 单步执行程序\\ \hline
 step & 进入所调用的函数内部, 查看执行情况\\ \hline
 whatis & 查看变量或函数类型, 调用格式为``whatis 变量名/函数名"\\ \hline
 ptype & 显示数据结构定义情况\\ \hline
 make & 编译程序\\ \hline
 quit & 退出GDB\\ \hline
\end{tabular}
\end{table}

%%%%%%%%%%%%%%%%%%%%%%%%%%%%%%%%%%%%%%%%%%
%%%%%%%%%%%%%%%%%%%%%%%%%%%%%%%%%%%%%%%%%%

\subsection{Linux开发环境}
\subsubsection{VIM}
要使用C语言源代码语法高亮, 需要配置文件 /etc/vim/vimrc, 加入代码 syntax on.
另外, 源代码文件的后缀名必须为 ``.c".

\subsubsection{Emacs}
\begin{shell}
 emacs helloworld.c // 启动Emacs, 并新建了文件helloworld.c
\end{shell}
在 Emacs 中编辑 helloworld 程序代码, 按组合键 \cmd{Alt+X}, 然后键入 compile 后按 Enter,
这时出现提示输入编译命令. 删除原有编译命令, 输入 \cmd{gcc -g -o helloworld helloworld.c}
后按 Enter 键, 完成编译. 按组合键 \cmd{Alt+X}, 然后输入 shell 后按 Enter 键, 
系统会弹出一个终端窗口, 在该窗口运行程序.

\subsubsection{集成开发环境}
集成开发环境\index{集成开发环境}是将一些开发工具集合到同一个操作界面的工具软件.
最初 Eclipse 用于开发 Java 语言程序, 但加入 CDT 插件后就能进行 C 和 C++ 语言程序开发.
Eclipse 卸载插件: 
\begin{verbatim}
 Help -> Install New Software -> already installed ...
\end{verbatim}
即可.
Kdevelop 是一个支持多种程序设计语言的集成开发环境.

%%%%%%%%%%%%%%%%%%%%%%%%%%%%%%%%%%%%%%%%%%
%%%%%%%%%%%%%%%%%%%%%%%%%%%%%%%%%%%%%%%%%%

\subsection{C 语言编程基础}
程序设计语言是用来描写计算机程序的逻辑语法结构. 高级语言的下一个发展目标是面向应用, 
自动生成算法. C 语言通常称为中级语言, 具有直接访问计算机底层资源的能力. 

从本质上来说, 数据类型是告诉编译器如何为数据分配内存, 内存的读取和写入方式却是相同的.

C 语言的另一个重要特点是它仅有32个关键字.

C 语言是结构化语言, 结构化语言所使用的设计方法为模块化设计方法, 每个子问题求解的步骤被定义为模块.

C 语言是程序员的语言, Linux 内核和大多数硬件的驱动程序是使用 C 语言设计的.

C 语言预处理命令以 ``\#" 开始.

从技术上讲, 主函数不是 C 语言的一个成分, 但它仍被看作是其中的一部分, 因此, main 不能用作变量名.

stdio.h 头文件存放的是输入输出相关的函数.

函数的集合称之为函数库, 遵守 ANSI C 标准的编译器所提供函数组成的函数库称之为 ANSI C 标准函数库.

编译器编译源代码时以函数为单位进行编译, 并记忆函数的名字.

通用建模语言(UML)?

C 语言程序中的变量在习惯上使用小写英文字母, 常量和其他用途的符号可用大写字母.
C 语言对大小写字母是敏感的, 关键字必须小写.

程序测试经常出现的问题有逻辑错误, 目标错误, 语法错误和编写错误.

\begin{table}[H]
\small % fontsize
\sffamily % font family = sans serif
\centering % alignment of tabular
\renewcommand{\arraystretch}{1.4} % larger cell height
\caption{C 语言关键字表}
\begin{tabular}{*{2}{c|c|c|c|c|c|c|c}}
 \hline
 auto & break & case & char & const & continue & default & do\\ \hline
 double & else & enum & extern & float & for & goto & if\\ \hline
 int & long & register & short & signed & sizeof & static & return\\ \hline
 struct & switch & typedef & union & unsigned & void & volatile & while\\ \hline
\end{tabular}
\end{table}

计算机程序的定义, 程序等于数据结构加算法. 数据结构是将事物抽象成为可运算的数据形式输入到计算机中,
算法是对这些数据计算的方法.

汇编语言等非结构化程序使用 goto 命令, 该命令能够使程序以行为单位跳转到任何位置. 代码中大量的 goto
命令使流程过于混乱, 源代码的维护难度极高.

C 语言3种基本的程序结构: 顺序结构, 选择结构和循环结构.

N-S 图描述算法, 适合于设计系统程序, 特别是分支间相互独立的程序.

PAD 图(Problem Analysis Diagram)是国际上专业的程序员广泛使用的一种程序流程表示法.

软件是程序与文档的总和. 程序设计只包含了程序设计和编码的反复迭代过程.

\subsection{终止编译命令}

\subsection{条件编译命令}

\subsection{修改行号命令}

\subsection{编译指示命令}

\subsection{预定义的宏名}
