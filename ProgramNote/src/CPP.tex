\chapter{C++}
\section{ACM}
ACM-ICPC: 在线评测系统, Online Judge, OJ

PKU: http://poj.org

USACO: 美国信息学奥林匹克, C, C++, Java, Python, http://train.usaco.org/usacogate

CII: 2500道

SGU: http://acm.sgu.ru

SPOJ: 波兰, http://www.spoj.pl/

网上比赛: GCJ: http://code.google.com/codejam

TopCoder: (TOC/TCHS)

Code forces

SCL: 相对较难实现的常用算法的代码整理, Standard Code Library, 标准代码库.
\section{线程池}
这里给出的线程池具有动态伸缩性, 能根据执行任务的轻重自动调整线程池中线程的数量.

为什么需要线程池, 创建和销毁线程的开销可以忽略. 线程池能够减少创建的线程个数. 
通常线程池所允许的并发线程是有上界的, 当并发的线程数超过上界, 则一部分线程会等待.

线程池采用预创建的技术, 在程序启动后创建一定数量($N$)的线程, 放入空闲队列. 
这些线程都处于阻塞状态, 不消耗CPU, 但占用较小的内存空间. 当任务到来时, 
缓冲池选择一个空闲线程运行任务. 当所有$N$个线程都在处理任务后, 
缓冲池自动创建一定数量的新线程, 用于处理更多任务. 当任务执行完, 线程不退出,
而是保持在池中等待下一次任务. 当系统比较空闲时, 大部分线程都处于暂停, 线程池自动销毁部分线程,
回收系统资源.

基于这种预创建技术, 线程池将线程创建和销毁分摊到任务上. 不过可能要考虑线程间同步带来的开销.

这里得线程池框架由五个重要部分组成 CThreadManage, CThreadPool, CThread, CJob, 
CWorkThread, 此外还有线程同步使用的类 CThreadMutex 和 CCondition.
CThreadMutex 用于线程间的互斥, CCondition 是条件变量的封装, 用于线程间的同步.
\section{排序算法}
\subsection{插入排序}
对待排序数组$a[0...n-1]$排序.
\inputcode[C]{code/sort/insert.c}
最差和平均时间复杂度是$O(n^2)$, 最优复杂度$O(n)$. 例如数组(5,1,4,2,8)经过插入排序的过程是
\begin{enumerate}[1). ]
 \item 插入1, (1,5,4,2,8)
 \item 插入4, (1,4,5,2,8)
 \item 插入2, (1,2,4,5,8)
 \item 插入8, (1,2,4,5,8)
\end{enumerate}
\subsection{冒泡排序}
对待排序数组$a[0...n-1]$排序.
\inputcode[C]{code/sort/buble.c}
时间复杂度$O(n^2)$, 每次排序选出一个最大值, 每两相邻的排序.

\section{使用CodeViz生成C/C++函数调用关系图}
首先安装 graphviz: 
\cmd*{sudo dnf install graphviz graphviz-devel graphviz-doc}

\begin{shell}
 graphviz : rich set of graph drawing tools
 graphviz-devel : transitional package for graphviz-devel rename
 graphviz-doc : additional documentation for graphviz
\end{shell}
然后安装相关库
\begin{shell}
sudo dnf install graphviz-guile graphviz-lua graphviz-ocaml graphviz-perl graphviz-php graphviz-python graphviz-ruby graphviz-tcl
\end{shell}
最后安装CodeViz: \cmd*{sudo dnf install codeviz}. 下载git项目: \cmd*{git clone https://github.com/petersenna/codeviz.git}, 
然后编译
\begin{shell}
 ./configure
 make
 sudo make install
\end{shell}


\newpage
\section{C++画图}
Linux下编译
\inputcode[C]{code/plot0001.c}
报错\cmd{undefined reference to `sqrt'}, 
编译时添加参数\cmd*{gcc -lm src.cpp -o obj}.

\newpage
\section{cuckoo}
Cuckoo算法
\inputcode[C]{code/cuckoo/nvromTest.c}
\inputcode[C]{code/cuckoo/cuckooFilter.h}
\inputcode[C]{code/cuckoo/cuckooFilter.c}
\inputcode[C]{code/cuckoo/mozilla-sha1/sha1.h}
\inputcode[C]{code/cuckoo/mozilla-sha1/sha1.c}

\newpage